\section{Installation}
This appendix describes how to build self-propelled instrumentation
from source code, which can be downloaded from \url{http://www.paradyn.org}
or \url{http://www.dyninst.org}.

Before starting to build self-propelled instrumentation, you have to make sure
that you have already installed and built Dyninst (refer Appendix D in Dyninst
Programming
Guide\footnote{\url{http://www.dyninst.org/sites/default/files/manuals/dyninst/dyninstProgGuide.pdf}
} for building Dyninst).

Building self-propelled instrumentation on Linux platforms is a very simple
three step process that involves: unpacking the self-propelled instrumentation
source, configuring and running the build.

\subsection{Getting the source}
Self-propelled instrumenation`s source code is packaged in \textit{tar.gz}
format. If your self-propelled instrumentation source tarball is called
\textit{src\_spi.tar.gz}, then you could extract it with the commands
\textit{gunzip src\_spi.tar.gz ; tar -xvf src\_spi.tar}. This will create a list
of directories and files.

\subsection{Configuration}
After unpacking, the next thing is to set SP\_DIR, DYNINST\_ROOT, PLATFORM and
DYNLINK variables in \textit{config.mk}.  DYNINST\_ROOT should be set to path of
the directory that contains subdirectories like dyninstAPI, parseAPI etc.,
i.e. within dyninst directory SP\_DIR should be set to the the path of the
current working directory (where self-propelled instrumentation is
installed).

DYNLINK should be set \textit{true} for building agent as a small shared library
that relies on other shared libraries. Otherwise, set DYNLINK as \textit{false}
for buidling a single huge shared library that static-linked all libraries

PLATFORM should be set to one of the following values depending upon what
operating system you are running on:
\begin{table}[h]
\begin{tabular}{l c l}
 i386-unknown-linux 2.4  & : & Linux 2.4/2.6 on an Intel x86 processor \\
 x86\_64-unknown-linux2.4&: &Linux 2.4/2.6 on an AMD-64/Intel x86-64  processor \\
\end{tabular}
\end{table}

Before building, you should also check whether LD\_LIBRARY\_PATH environment
variable is set. If you are using bash shell, then open ~/.bashrc file and check
if LD\_LIBRARY\_PATH is already present. If not, then LD\_LIBRARY\_PATH variable
should be set in a way that it includes {\$DYNINST\_ROOT}/lib directories.  If
you are using C shell, then do the above mentioned tasks in ~/.cshrc file.

If you want to use inter-process propagation, then you should also set the
environment variable SP\_AGENT\_DIR to be the full path of the agent shared
library that will be injected on every machine involved.
 
\subsection{Building}
Once config.mk is set, you are ready to build self-propelled
instrumentation. Move to PLATFORM directory and execute the command
\textit{make}. This will build libagent.so, injector, test-cases and
test-programs. Other make commands for custom building are as follows:

\begin{itemize}
\item make spi: build injector and libagent.so.
\item make injector\_exe: build injector.
\item make agent\_lib: build libagent.so.
\item make unittest: build unittests.
\item make mutatee: build simple mutatees.
\item make external\_mutatee: build real world mutatees, including unix core
  utilities and gcc.
\item make test: equivalent to make unittest + make mutatee + make
  external\_mutatee
\item make clean\_test: clean test stuffs.
\item make clean: only clean core self-propelled stuffs, excluding dependency.
\item make clean\_all: clean everything, including dependency.
\item make clean\_objs: clean core self-propelled objs.
\end{itemize}

That is it! Now, you are all set to use Self propelled intrumentation.
