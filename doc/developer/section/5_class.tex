\section{Class Reference}

% In both user and developer manual
\subsection{Public Interface}
\subsubsection{Class Agent}
\textbf{Declared in}: src/agent/agent.h

\begin{apient}
void SetInitEntry(string);
\end{apient}
\apidesc{

Sets the entry payload function name. By default, the entry payload function is
``default_entry'', which simply prints out all executed functions' name.

}

\begin{apient}
void SetInitExit(string);
\end{apient}
\apidesc{

Sets the exit payload function name. By default, there is not exit payload
function.

}

\begin{apient}
void SetLibrariesToInstrument(const StringSet& libs);
\end{apient}
\apidesc{

Sets the set of names of libraries loaded in the application process that will
be instrumented. By default, we only instrument the executable binary code and
don't instrument loaded shared libraries.

}

\begin{apient}
void SetFuncsNotToInstrument(const StringSet& funcs);
\end{apient}
\apidesc{
Sets the set of names of functions in the application process that will
NOT be instrumented. By default, we instrument all functions.
}

\begin{apient}
void EnableParseOnly(const bool yes_or_no);
\end{apient}
\apidesc{

Enables or disables ParseOnly option. If yes\_or\_no is true, then after parsing
the binary code, we don't do instrumentation; otherwise, instrumentation is
conducted after parsing. By default, ParseOnly option is disabled.

}

\begin{apient}
void EnableIpc(const bool yes_or_no);
\end{apient}
\apidesc{

Enable or disable inter-process instrumentation propagation. By default,
inter-process instrumentation propagation is disabled.

}

\begin{apient}
void EnableHandleDlopen(const bool yes_or_no);
\end{apient}
\apidesc{

Enable or disable propagating instrumentation to library loaded via dlopen. By
default, it is disabled.

}

\begin{apient}
void EnableMultithread(const bool yes_or_no);
\end{apient}
\apidesc{

Enable or disable propagating instrumentation across thread. By default, it is
disabled.

}

\begin{apient}
void Go();
\end{apient}
\apidesc{

Starts initial instrumentation.

}

\subsubsection{Class SpPoint}
\textbf{Declared in}: src/agent/patchapi/point.h

\begin{apient}
bool tailcall();
\end{apient}
\apidesc{

Indicates whether or not the function call at this point is a tail call.

}

\begin{apient}
SpBlock* GetBlock() const;
\end{apient}
\apidesc{

Returns the call block at this point.

}

\begin{apient}
SpObject* GetObject() const;
\end{apient}
\apidesc{
Returns the binary object at this point.
}


\subsubsection{Class SpObject}
\textbf{Declared in}: src/agent/patchapi/object.h

\begin{apient}
std::string name() const;
\end{apient}
\apidesc{
Returns the binary object's name.
}

This class inherits PatchAPI::PatchObject. Please refer to PatchAPI document for
the complete list of methods.

\subsubsection{Class SpFunction}
\textbf{Declared in}: src/agent/patchapi/cfg.h

\begin{apient}
SpObject* GetObject() const;
\end{apient}
\apidesc{
Returns the binary object containing this function.
}

\begin{apient}
std::string GetMangledName();
\end{apient}
\apidesc{
Returns the mangled name of this function.
}

\begin{apient}
std::string GetPrettyName();
\end{apient}
\apidesc{
Returns the demangled name of this function.
}

\begin{apient}
std::string name();
\end{apient}
\apidesc{
Returns the demangled name of this function, the same as calling GetPrettyName().
}

This class inherits PatchAPI::PatchFunction. Please refer to PatchAPI document
for the complete list of methods.

\subsubsection{Class SpBlock}
\textbf{Declared in}: src/agent/patchapi/cfg.h

\begin{apient}
SpObject* GetObject() const;
\end{apient}
\apidesc{
Returns the binary object containing this block.
}

\begin{apient}
in::Instruction::Ptr orig_call_insn() const;
\end{apient}
\apidesc{

Returns an InstructionAPI::Instruction instance of the call instruction in this
block.

}

This class inherits PatchAPI::PatchBlock. Please refer to PatchAPI document for
the complete list of methods.

\subsubsection{Class SpEdge}
\textbf{Declared in}: src/agent/patchapi/cfg.h

This class inherits PatchAPI::PatchEdge. Please refer to PatchAPI document for
the complete list of methods.

\subsubsection{Utility Functions}
\textbf{Declared in}: src/agent/payload.h

Utility functions are used when writing payload functions.

\begin{apient}
SpFunction* Callee(SpPoint* pt);
\end{apient}
\apidesc{
Returns an instance of SpFunction for current function call. 
The argument \emph{pt} represents the instrumentation point for current function
call.
If it fails, it returns NULL.
}

\begin{apient}
bool IsInstrumentable(SpPoint* pt);
\end{apient}
\apidesc{
Indicates whether or not the point is instrumentable.
}

\begin{apient}
void Propel(SpPoint* pt);
\end{apient}
\apidesc{

Propagates instrumentation to callees of the function called at the specified
point {\em pt}.

}

\begin{apient}
bool IsIpcWrite(SpPoint* pt); 
bool IsIpcRead(SpPoint* pt); 
\end{apient}
\apidesc{

Indicates whether or not the function called at the specified point is a
inter-process communication write / read function.

}

\begin{apient}
struct ArgumentHandle;
void* PopArgument(SpPoint* pt,
                  ArgumentHandle* h,
                  size_t size);
\end{apient}
\apidesc{

Gets a pointer to an argument of the function call at the specified point {\em
  pt}.
All arguments passed to the function call are in a stack associated with the
ArgumentHandle structure {\em h}. 
Leftmost argument is at the top of the stack.
The {\em size} parameter specifies the size of the argument that is about to be
popped.

}

\begin{apient}
long ReturnValue(SpPoint* pt);
\end{apient}
\apidesc{
Returns the return value of the function call at the specified point {\em pt}.
}

% In developer manual only
\subsection{Private Interface}
\subsubsection{Class SpParser}
\textbf{Declared in}: src/agent/parser.h

\begin{apient}
static ptr Create();
\end{apient}
\apidesc{
}

\begin{apient}
SpObject* exe() const;
\end{apient}
\apidesc{
Returns the executable's binary object.
}

\begin{apient}
PatchAPI::PatchMgrPtr Parse();
\end{apient}
\apidesc{
}

\begin{apient}
PatchAPI::PatchMgrPtr Parse();
\end{apient}
\apidesc{
}

\begin{apient}
string agent_name()
\end{apient}
\apidesc{
}

\begin{apient}
bool injected()
\end{apient}
\apidesc{
}

\begin{apient}
void GetFrame(long* pc, long* sp, long* bp)
\end{apient}
\apidesc{
}

\begin{apient}
SpFunction* FindFunction(dt::Address absolute_addr);
\end{apient}
\apidesc{
}

\begin{apient}
SpFunction* FindFunction(string func_name_without_path);
\end{apient}
\apidesc{
}

\begin{apient}
bool FindFunction(string func_name_without_path, FuncSet* found_funcs);
\end{apient}
\apidesc{
}

\begin{apient}
SpFunction* callee(SpPoint* point, bool parse_indirect = false);
\end{apient}
\apidesc{
}

\begin{apient}
string DumpInsns(void* addr, size_t size);
\end{apient}
\apidesc{
}

\begin{apient}
static bool ParseDlExit(SpPoint* pt);
\end{apient}
\apidesc{
}

\subsubsection{Class SpContext}
\textbf{Declared in}: src/agent/context.h

\begin{apient}
static SpContext* Create();
\end{apient}
\apidesc{
}

\begin{apient}
string init_entry_name() const
\end{apient}
\apidesc{
}

\begin{apient}
string init_exit_name() const
\end{apient}
\apidesc{
}

\begin{apient}
void GetCallStack(FuncSet* func_set);
\end{apient}
\apidesc{
}

\begin{apient}
void GetCallStack(FuncSet* func_set);
\end{apient}
\apidesc{
}

\begin{apient}
bool IsMultithreadEnabled() const;
\end{apient}
\apidesc{
}

\begin{apient}
bool IsHandleDlopenEnabled() const;
\end{apient}
\apidesc{
}

\begin{apient}
bool IsDirectcallOnlyEnabled() const;
\end{apient}
\apidesc{
}

\begin{apient}
SpPropeller::ptr init_propeller() const;
\end{apient}
\apidesc{
}

\begin{apient}
PayloadFunc init_entry() const;
PayloadFunc init_exit() const;
\end{apient}
\apidesc{
}

\begin{apient}
PayloadFunc wrapper_exit() const;
PayloadFunc wrapper_entry() const;
\end{apient}
\apidesc{
}

\subsubsection{Class SpEvent}
\textbf{Declared in}: src/agent/event.h

\begin{apient}
static ptr Create();
\end{apient}
\apidesc{
}

\begin{apient}
virtual void RegisterEvent();
\end{apient}
\apidesc{
}

\subsubsection{Class AsyncEvent}
\textbf{Declared in}: src/agent/event.h

\begin{apient}
static ptr Create(int signum = SIGALRM,
                  int sec = 5);
\end{apient}
\apidesc{
}

\subsubsection{Class SyncEvent}
\textbf{Declared in}: src/agent/event.h

\begin{apient}
static ptr Create();
\end{apient}
\apidesc{
}

\subsubsection{Class FuncEvent}
\textbf{Declared in}: src/agent/event.h

\begin{apient}
static ptr Create(StringSet& funcs);
\end{apient}
\apidesc{
}

\subsubsection{Class CallEvent}
\textbf{Declared in}: src/agent/event.h

\begin{apient}
static ptr Create(StringSet& funcs);
\end{apient}
\apidesc{
}

\subsubsection{Class CombEvent}
\textbf{Declared in}: src/agent/event.h

\begin{apient}
static ptr Create(EventSet& events);
\end{apient}
\apidesc{
}

\subsubsection{Class SpPropeller}
\textbf{Declared in}: src/agent/propeller.h

\begin{apient}
static ptr Create();
\end{apient}
\apidesc{
}

\begin{apient}
bool go(SpFunction* func,
        PayloadFunc entry,
        PayloadFunc exit,
        SpPoint* pt = NULL,
        StringSet* inst_calls = NULL);
\end{apient}
\apidesc{
}

\subsubsection{Class SpSnippet}
\textbf{Declared in}: src/agent/snippet.h

\begin{apient}
static ptr create(SpFunction* f,
                  SpPoint* pt,
                  PayloadFunc entry,
                  PayloadFunc exit);
\end{apient}
\apidesc{
}

\begin{apient}
char* BuildBlob(const size_t est_size,
                const bool reloc = false);
\end{apient}
\apidesc{
}

\begin{apient}
size_t GetBlobSize() const;
\end{apient}
\apidesc{
}

\begin{apient}
SpBlock* FindSpringboard();
\end{apient}
\apidesc{
}

\begin{apient}
char* RelocateSpring(SpBlock* spring_blk);
\end{apient}
\apidesc{
}

\begin{apient}
size_t GetRelocSpringSize() const;
\end{apient}
\apidesc{
}

\begin{apient}
dt::Address GetSavedReg(dt::MachRegister reg);
\end{apient}
\apidesc{
}

\begin{apient}
long GetRetVal();
\end{apient}
\apidesc{
}

\begin{apient}
void* PopArgument(ArgumentHandle* h, size_t size);
\end{apient}
\apidesc{
}

\begin{apient}
PayloadFunc entry() const;
\end{apient}
\apidesc{
}

\begin{apient}
PayloadFunc exit() const;
\end{apient}
\apidesc{
}

\begin{apient}
SpPoint* point() const;
\end{apient}
\apidesc{
}


\subsubsection{Class SpAddrSpace}
\textbf{Declared in}: src/agent/patchapi/addr\_space.h

\begin{apient}
bool SetMemoryPermission(dt::Address addr,
                         size_t length,
                         int perm);
\end{apient}
\apidesc{
}

\begin{apient}
void InitMemoryAllocator();
\end{apient}
\apidesc{
}

\subsubsection{Inst Workers}

\textbf{Class InstWorkerDelegate}

\textbf{Declared in}: src/agent/inst\_workers/inst_worker_delegate.h

\begin{apient}
virtual bool run(SpPoint* pt);
\end{apient}
\apidesc{
Instrument a point.
}

\begin{apient}
virtual bool undo(SpPoint* pt);
\end{apient}
\apidesc{
Uninstrument a point.
}

\begin{apient}
virtual bool save(SpPoint* pt);
\end{apient}
\apidesc{
Save code that will be modified for a point.
}

\begin{apient}
virtual InstallMethod install_method() const;
\end{apient}
\apidesc{
Returns the method of instrumentation installation.
}

\textbf{Class RelocCallInsnWorker}

\textbf{Declared in}: src/agent/inst\_workers/callinsn_worker_impl.h

\textbf{Class RelocCallBlockWorker}

\textbf{Declared in}: src/agent/inst\_workers/callblk_worker_impl.h

\textbf{Class SpringboardWorker}

\textbf{Declared in}: src/agent/inst\_workers/spring_worker_impl.h

\textbf{Class TrapWorker}

\textbf{Declared in}: src/agent/inst\_workers/trap_worker_impl.h


\subsubsection{Class SpThreadMgr}
\textbf{Declared in}: src/agent/thread\_mgr.h

\begin{apient}
static bool BeforeEntry(SpPoint*);
\end{apient}
\apidesc{
}

\subsubsection{Class IpcMgr}
\textbf{Declared in}: src/agent/ipc/ipc\_mgr.h

\begin{apient}
SpChannel* GetChannel(int fd,
                      ChannelRW rw);
\end{apient}
\apidesc{
}

\begin{apient}
SpIpcWorkerDelegate* GetWorker(int fd);
\end{apient}
\apidesc{
}

\begin{apient}
void GetWriteParam(SpPoint* pt,
                   int* fd_out,
                   void** buf_out,
                   char* c_out,
                   size_t* size_out,
                   sockaddr** sa_out);
\end{apient}
\apidesc{
}

\begin{apient}
void GetReadParam(SpPoint* pt,
                  int* fd_out,
                  void** buf_out,
                  size_t* size_out);
\end{apient}
\apidesc{
}

\begin{apient}
bool is_fork(const char* f);
\end{apient}
\apidesc{
}

\begin{apient}
bool is_popen(const char* f);
\end{apient}
\apidesc{
}

\begin{apient}
char start_tracing(int fd);
\end{apient}
\apidesc{
}

\begin{apient}
static bool BeforeEntry(SpPoint*);
\end{apient}
\apidesc{
}

\begin{apient}
static bool BeforeExit(SpPoint*);
\end{apient}
\apidesc{
}

\begin{apient}
SpPipeWorker* pipe_worker() const;
\end{apient}
\apidesc{
}

\begin{apient}
SpTcpWorker* tcp_worker() const;
\end{apient}
\apidesc{
}

\begin{apient}
SpUdpWorker* udp_worker() const;
\end{apient}
\apidesc{
}

\subsubsection{Class SpChannel}
\textbf{Declared in}: src/agent/ipc/channel.h

\subsubsection{IPC Workers}
\textbf{Declared in}: src/agent/ipc/ipc\_workers/

\textbf{Class SpIpcWorkerDelegate}

\textbf{Declared in}: src/agent/ipc/ipc\_workers/ipc\_worker\_delegate.h

\begin{apient}
virtual void SetStartTracing(char yes_or_no,
                             SpChannel* c);
\end{apient}
\apidesc{
}

\begin{apient}
virtual void SetStartTracing(char yes_or_no);
\end{apient}
\apidesc{
}

\begin{apient}
virtual char start_tracing(int fd);
\end{apient}
\apidesc{
}

\begin{apient}
virtual bool Inject(SpChannel*, char* agent_path = NULL);
\end{apient}
\apidesc{
}

\begin{apient}
virtual SpChannel* GetChannel(int fd,
                              ChannelRW rw,
                              void* arg = NULL);
\end{apient}
\apidesc{
}

\textbf{Class SpPipeWorker}

\textbf{Declared in}: src/agent/ipc/ipc\_workers/pipe\_worker\_impl.h

\textbf{Class SpTcpWorker}

\textbf{Declared in}: src/agent/ipc/ipc\_workers/tcp\_worker\_impl.h

\textbf{Class SpUdpWorker}

\textbf{Declared in}: src/agent/ipc/ipc\_workers/udp\_worker\_impl.h
