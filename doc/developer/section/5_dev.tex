% In developer manual only
\subsection{Class SpInjector}
\textbf{Declared in}: src/injector/injector.h

\begin{apient}
static ptr Create(Dyninst::PID pid);
\end{apient}
\apidesc{
Creates a Boost shared pointer pointing to a SpInjector instance.
The parameter {\em pid} specifies the pid of the remote process that will be
injected.

}

\begin{apient}
bool Inject(const char* lib_name);
\end{apient}
\apidesc{
Injects a library with {\em lib\_name}.
}


\subsection{Class SpParser}
\textbf{Declared in}: src/agent/parser.h

\begin{apient}
static ptr Create();
\end{apient}
\apidesc{
Creates a Boost shared pointer pointing to a SpParser instance.
}

\begin{apient}
SpObject* exe() const;
\end{apient}
\apidesc{
Returns the executable's binary object.
}

\begin{apient}
PatchAPI::PatchMgrPtr Parse();
\end{apient}
\apidesc{ The major parsing method, which builds CFG data structures to assist
  instrumentation.  }

\begin{apient}
string agent_name()
\end{apient}
\apidesc{
Returns the agent library's full path name.
}

\begin{apient}
bool injected()
\end{apient}
\apidesc{

  Indicates whether or not this agent library is injected in the middle of
  running current process.

}

\begin{apient}
void GetFrame(long* pc, long* sp, long* bp)
\end{apient}
\apidesc{

  Get the values of three registers (pc, sp, and bp) of the call frame where the
  agent is injected. The three parameters are output parameters.

}

\begin{apient}
SpFunction* FindFunction(Dyninst::Address absolute_addr);
\end{apient}
\apidesc{

  Returns the function that contains the address {\em absolute\_addr}.
  It returns NULL if no function is found.
}

\begin{apient}
SpFunction* FindFunction(string mangled_name);
\end{apient}
\apidesc{

  Returns one of the functions that have the mangled function name.
  It returns NULL if no function is found.
}

\begin{apient}
bool FindFunction(string func_name, FuncSet* found_funcs);
\end{apient}
\apidesc{

  Get all the functions that have the demangled function name.

}


\begin{apient}
SpFunction* callee(SpPoint* point, bool parse_indirect = false);
\end{apient}
\apidesc{

  Returns the callee at the given call {\em point}.
  If {\em parse\_indirect} is false and the call instruction is an indirect call
  instruction, then we skip parsing it and return NULL immediately.
  If the callee is not found, NULL is returned.
}

\begin{apient}
string DumpInsns(void* addr, size_t size);
\end{apient}
\apidesc{

  Returns a nice textual representation of all instructions at the memory area of [{\em
    addr}, {\em addr} + size].

}

\begin{apient}
static bool ParseDlExit(SpPoint* pt);
\end{apient}
\apidesc{

  Detects the dlopen function call, and parses the library that is just loaded.
  It is invoked immediately after the dlopen function call is returned.
}

\subsection{Class SpContext}
\textbf{Declared in}: src/agent/context.h

\begin{apient}
static SpContext* Create();
\end{apient}
\apidesc{
Creates a Boost shared pointer pointing to a SpContext instance.

}

\begin{apient}
string init_entry_name() const
\end{apient}
\apidesc{
  Returns initial entry payload function's demangled name.
}

\begin{apient}
string init_exit_name() const
\end{apient}
\apidesc{
  Returns initial exit payload function's demangled name.
}

\begin{apient}
typedef std::set<SpFunction*> FuncSet;
void GetCallStack(FuncSet* func_set);
\end{apient}
\apidesc{
  Get a set of functions in the call stack, when the agent library is injected.
  {\em func\_set} is the output parameter.
}


\begin{apient}
bool IsMultithreadEnabled() const;
\end{apient}
\apidesc{
  Indicates whether instrumentation propagation across thread boundary is supported.
}

\begin{apient}
bool IsHandleDlopenEnabled() const;
\end{apient}
\apidesc{
  Indicates whether parsing newly loaded shared library is supported.
}

\begin{apient}
bool IsDirectcallOnlyEnabled() const;
\end{apient}
\apidesc{

  Indicates whether or not self-propelled instrumentation only instruments
  direct function call.

}

\begin{apient}
SpPropeller::ptr init_propeller() const;
\end{apient}
\apidesc{
  Returns an instance of SpPropeller.
}

\begin{apient}
typedef void* PayloadFunc;
PayloadFunc init_entry() const;
PayloadFunc init_exit() const;
\end{apient}
\apidesc{

  Returns the start address of user-provided entry payload function or exit
  payload function.

}

\begin{apient}
PayloadFunc wrapper_exit() const;
PayloadFunc wrapper_entry() const;
\end{apient}
\apidesc{

  Returns the start address of the wrapper of entry payload function or exit
  payload function.
  
}

\subsection{Events}
\textbf{Declared in}: src/agent/event.h

This is a collection of events that trigger initial instrumentation.

\textbf{Class SpEvent}

It is the base class of all event classes. This event does not trigger any
instrumentation.

\begin{apient}
static ptr Create();
\end{apient}
\apidesc{
Creates a Boost shared pointer pointing to a SpEvent instance.

}

\begin{apient}
virtual void RegisterEvent();
\end{apient}
\apidesc{
It registers the event to agent. This function is called in Agent::Go().
}

\textbf{Class AsyncEvent}

It is the asynchronous event that will happen when specified signal occurs.

\begin{apient}
static ptr Create(int signum = SIGALRM,
                  int sec = 5);
\end{apient}
\apidesc{
Creates a Boost shared pointer pointing to an AsyncEvent instance.

The parameter {\em signum} specifies the signal number, and the parameter {\em
  sec} is only effective when signum is SIGALRM.

}

\textbf{Class SyncEvent}

It is the synchronous event that will do instrumentation immediately.
If the agent library is loaded via LD\_PRELOAD, then it instruments main()
function immediately.
If the agent library is injected via injector, then it instruments all functions
in the call stack when the agent library is loaded.

\begin{apient}
static ptr Create();
\end{apient}
\apidesc{
Creates a Boost shared pointer pointing to a SyncEvent instance.

}


\textbf{Class FuncEvent}

It instruments all of the specified functions immediately.

\begin{apient}
static ptr Create(StringSet& funcs);
\end{apient}
\apidesc{
Creates a Boost shared pointer pointing to a FuncEvent instance.
The parameter {\em funcs} specifies a set of demangled names of functions that
are going to be instrumented.  
}

\textbf{Class CallEvent}

It installs instrumentation at all call sites of specified functions immediately.

\begin{apient}
static ptr Create(StringSet& funcs);
\end{apient}
\apidesc{
Creates a Boost shared pointer pointing to a CallEvent instance.
The parameter {\em funcs} specifies a set of demangled names of function calls
that are going to be instrumented.

}

\textbf{Class CombEvent}

It is a compound event that can combine all the above events.

\begin{apient}
static ptr Create(EventSet& events);
\end{apient}
\apidesc{
Creates a Boost shared pointer pointing to a CombEvent instance.
The parameter {\em events} specifies a set of events.

}

\subsection{Class SpPropeller}
\textbf{Declared in}: src/agent/propeller.h

\begin{apient}
static ptr Create();
\end{apient}
\apidesc{
Creates a Boost shared pointer pointing to a SpPropeller instance.

}

\begin{apient}
bool go(SpFunction* func,
        PayloadFunc entry,
        PayloadFunc exit,
        SpPoint* pt = NULL,
        StringSet* inst_calls = NULL);
\end{apient}
\apidesc{

  It propagates instrumentation to callees of specified function {\em func}.
  The parameters {\em entry} and {\em exit} specify the payload functions to be
  installed for those callees.  
  The parameter {\em point} is the point where {\em func} is invoked.
  The parameter {\em inst\_calls} specifies all function calls that need to be
  instrumented, which is used only in CallEvent.

}




\subsection{Class SpSnippet}
\textbf{Declared in}: src/agent/snippet.h


\begin{apient}
char* BuildBlob(const size_t est_size,
                const bool reloc = false);
\end{apient}
\apidesc{

  Returns a pointer to a blob of code snippet, which contains relocated code and
  invocations to payload functions.
  The parameter {\em est\_size} specifies the estimate size to allocate the buffer
  for the code snippet.  
  The parameter {\em reloc} indicates whether or not we need to relocate a call
  block.

}

\begin{apient}
size_t GetBlobSize() const;
\end{apient}
\apidesc{
  Returns the code snippet's size.
}

\begin{apient}
SpBlock* FindSpringboard();
\end{apient}
\apidesc{
  Returns a springboard block.
}

\begin{apient}
char* RelocateSpring(SpBlock* spring_blk);
\end{apient}
\apidesc{

  Relocates the basic block {\em spring\_blk} that is used as springboard block,
  and returns the buffer containing the relocated springboard block.

}

\begin{apient}
size_t GetRelocSpringSize() const;
\end{apient}
\apidesc{
  Returns the relocated springboard block size.
}

\begin{apient}
Dyninst::Address GetSavedReg(Dyninst::MachRegister reg);
\end{apient}
\apidesc{
  Returns the value of a saved register {\em reg}.
}

\begin{apient}
long GetRetVal();
\end{apient}
\apidesc{
  Returns the return value of current function call.
}

\begin{apient}
void* PopArgument(ArgumentHandle* h, size_t size);
\end{apient}
\apidesc{
  Pops an parameter of current function call.
}

\begin{apient}
PayloadFunc entry() const;
\end{apient}
\apidesc{
  Returns entry payload function's address.
}

\begin{apient}
PayloadFunc exit() const;
\end{apient}
\apidesc{
  Returns exit payload function's address.
}

\begin{apient}
SpPoint* point() const;
\end{apient}
\apidesc{
  Returns the function call point where this code snippet is installed.
}

