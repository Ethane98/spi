\section{Abstraction}
Self-propelled instrumentation has two major components, {\em Agent} that is a
shared library injected into an application process's address space, and {\em
  Injector} that injects {\em Agent}.
The following subsections describe the lower level components in Agent and
Injector in details.

\subsection{Agent}

% For user / developer manual
\subsubsection{User-level Abstractions}
\begin{itemize}
\item Agent. It manages the configuration and does instrumentation. An Agent
  instance is created in the init function of the {\em Agent} shared library.
\item Event. It specifies what kind of initial instrumentation should be done
  once the {\em Agent} shared library is loaded. 
  Currently, there are three types of Event: 1) instrumenting all callees in {\em
  main} function right away; 2) instrumenting all callees of specified functions
  right away; 3) instrumenting specified function calls right away.
\item Payload function. It contains user-specified code. 
  From user's perspective, a payload function will be invoke before or after each
  function call in the process.
  There are two types of payload functions, {\em entry payload} that is invoked
  before each function call and {\em exit payload} that is invoked after each
  function call.
\item Point. It represents an instrumentation point at current function call and
  is used in Payload function.
\item Control Flow Graph (CFG) structures. CFG structures include Object,
  Function, Block, and Edge. An Object represents a binary file (i.e., an
  executable or a shared library), and contains a set of functions. A Function
  contains a set of Blocks. A Block is a basic block. An Edge connects two
  Blocks. Users can get related CFG structures of current function call from
  Point.

% For developer / Use PatchAPI
\subsubsection{Developer-level Abstractions}

\item AddressSpace. It represents the address space of the process. It contains
  a set of Objects in the process. Also, it implements some memory management
  primitives used by the instrumentation engine.
\item Parser. It represents a binary code parser that parses binary code into
  structural CFG structures, i.e., Object, Function, Block, and Edge.
\item Propeller. It manages intra-process instrumentation propagation, where it
  finds function call Points inside current function and uses Instrumenter to
  insert Snippets at these points.
\item Snippet. It represents a patch area that contains function calls to the
  Payload function and the relocated function call or the relocated call block.
\item Instrumenter. It is the instrumentation engine that uses a set of
  Instrumentation Workers to insert Snippets to function call points.
\item Instrumentation Worker. It represents a mechanism of installing
  instrumentation. Currently, four types of Instrumentation Workers are
  implemented: 1) relocating original function call instruction; 2) relocating
  original call block; 3) relocating nearby large springboard block; 4) using
  trap instruction.
\item IpcMgr. It manages inter-process instrumentation propagation by creating
  Channels and using IPC Workers.
\item Channel. It represents a unidirectional communication channel, containing
  local process name and remote process name.
\item IPC Worker. It implements inter-process instrumentation propagation for a
  particular IPC mechanism (e.g., TCP, UDP, pipe).
\end{itemize}

\subsection{Injector}
% For user / developer manual
% Two types of injection: process start and hijack
\subsubsection{User-level Abstractions}

Injector is provided as a command. 
There are two types of injections.
One is to inject the {\em Agent} shared library at the very beginning of a
process.
The other is to inject the {\em Agent} in the middle of a running process.

% For developer manual
% Two types of injection: process start and hijack
\subsubsection{Developer-level Abstractions}

The first type of Injector relies on dynamic linker (i.e., setting the
environment variable LD\_PRELOAD to the path of an {\em Agent} shared library).
The second type uses ProcControlAPI to force an application process to invoke
functions in the dlopen family.
