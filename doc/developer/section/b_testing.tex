\section{Testing and Debugging}
This section covers the testing framework and some debugging tricks of
self-propelled instrumentation.
\subsection{Testing}
We rely on Google C++ Testing Framework for testing, whose source code is
included in self-propelled instrumentation's project directory, that is,
\$SP\_DIR/src/test/gmock.

There are two types of tests in the project, unit tests and system tests.

Each source code file is associated with a unit test suite, consisting of a set
of test cases to test component units.
The unit test suite code is located within the same folder of the source code it
tests (mostly in \$SP\_DIR/src/agent, \$SP\_DIR/src/injector, and
\$SP\_DIR/src/common), and the unit test suite file name has a suffix
``\_unittest''.
For example, if a source code file name is agent.cc, then its unit test suite
file name is agent\_unittest.cc.

System tests are located in \$SP\_DIR/src/test, which test the complete system.
There are application programs for testing.
The application programs that are written by self-propelled instrumentation
developers are located in \$SP\_DIR/src/test/mutatee.
Those that are real external programs (e.g., unix core utilities) are downloaded
by the script in \$SP\_DIR/scripts/build\_mutatees.sh that is invoked implicitly
by Makefile.

To run tests, the current working directory need to be the building directory
\$SP\_DIR/\$PLATFORM, and we invoke test/\$TEST\_NAME.exe to run a test.
For example, we can run test/coreutils\_systest.exe to run the system test for
instrumenting unix core utilities, or run test/injector\_unittest.exe to run the
system test for injector.

\subsection{Debugging}

We provide several useful environment variables to assist debugging.
\begin{itemize}
\item SP\_DEBUG: If this environment variable is set, then the execution of
  self-propelled instrumentation would print out detailed debug information to
  the stderr.
\item SP\_COREDUMP: If this environment variable is set, then it enables core
  dump when segfault happens.
\item SP\_DEBUG: enables printing out debugging messages
\item SP\_TEST\_RELOCINSN: If this environment variable is set, then self-propelled
  instrumentation only uses instruction relocation instrumentation worker
\item SP\_TEST\_RELOCBLK: If this environment variable is set, then self-propelled
  instrumentation only uses call block relocation instrumentation worker
\item SP\_TEST\_SPRING: If this environment variable is set, then self-propelled
  instrumentation only uses sprint block instrumentation worker
\item SP\_TEST\_TRAP: If this environment variable is set, then self-propelled
  instrumentation only uses trap instrumentation worker.
\item SP\_NO\_TAILCALL: If this environment variable is set, then self-propelled
  instrumentation will not instrument tail calls.
\item SP\_LIBC\_MALLOC: If this environment variable is set, then on x86\_64,
  self-propelled instrumentation will always use libc malloc, instead of using
  our customized memory allocator to find nearby buffers.
\item SP\_NO\_LIBC\_MALLOC: If this environment variable is set, then on
  x86\_64, self-propelled instrumentation will not use libc malloc when no any
  nearby buffer is found.
\end{itemize}
